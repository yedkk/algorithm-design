\documentclass[11pt]{article}
\usepackage{graphicx}
\graphicspath{ {C:/Users/yedkk/Desktop/CS465/HW1} }
\begin{document}
\section{Homework 2}
Name: Kangdong Yuan
	
\subsection{probelm1}
a).I did not work in a group.
\\b).I did not consult without anyone my group members
\\c).I did not consult any non-class materials.

\subsection{problem2}
a).$f=\Omega(g)$\\ both f and g are the power of n, but $1.5>1.3$, so $\lim _{n\to \infty }\left(\frac{f\left(n\right)}{g\left(n\right)}\right)=\infty $.
\\b).$f=\Theta (g)$
\\ because $f(n)=0.5*2^{n}=O(2^n)=g(n)$ for $\lim _{n\to \infty }\left(\frac{f\left(n\right)}{g\left(n\right)}\right)=0.5$.
\\c).$f=\Omega (g)$
\\if base of log is 2, for $n\geq 2.23$, $1.3log(n)>1.5$, so the $n^{1.3log(n)}>n^{1.5}$. $\lim _{n\to \infty }\left(\frac{f\left(n\right)}{g\left(n\right)}\right)=\infty $.
\\d). $f=\Omega(g)$ 
\\ the $f=\Theta(3^n)$ and $g=\Theta(2^n)$. The base of n are $3>2$. So for $\lim _{n\to \infty }\left(\frac{f\left(n\right)}{g\left(n\right)}\right)=\infty $.
\\e).$f=O(g)$
\\ take log on f and g, $log(f)=100log(log(n))<0.1log(n)=log(g)$, so $\lim _{n\to \infty }\left(\frac{f\left(n\right)}{g\left(n\right)}\right)=0 $.
\\f). $f=\Omega (g)$
\\take log for f and g, $log(f(n))=log(n)>loglog(n)*log(log(n))$, so $\lim _{n\to \infty }\left(\frac{f\left(n\right)}{g\left(n\right)}\right)=\infty $.
\\g). $f=O(g)$
\\take log for f and g, $log(f(n))=log(n)<\prod _{i=0}^n\:log\left(i\right)=log(g(n))$,
\\so $\lim _{n\to \infty }\left(\frac{f\left(n\right)}{g\left(n\right)}\right)=0 $.
\\h).$f=O(g)$
\\$f(n)=nlog(e)$, for $n>e$, the $log(e)<log(n)$. 
\\So for $n>1$, $f(n)=nlog(e)<nlog(n)=g(n)$. 
\\Thus $\lim _{n\to \infty }\left(\frac{f\left(n\right)}{g\left(n\right)}\right)=0 $.
\\i).$f=\Theta(g)$
\\$f(n)=n+log(n)=O(n)=n+(log(n))^2=g(n)$ for all $n>1$.
\\Thus $\lim _{n\to \infty }\left(\frac{f\left(n\right)}{g\left(n\right)}\right)=c, \ c \ is \ a \ constant \ number $
\\j).$f=\Theta(g)$
\\$f(n)=5n+\sqrt{n}=O(n)=log(n)+n=g(n)$ for all $n>1$.
\\Thus $\lim _{n\to \infty }\left(\frac{f\left(n\right)}{g\left(n\right)}\right)=c, \ c \ is \ a \ constant \ number $

\subsection{problem3}
a). There are $n+1$ items in polynomial. The sum of $n+1$ items take n steps. The multiplications in i items is $i+1$, the total multiplications is $\sum _{i=0}^n\:i+1=1+\frac{n^2+3n}{2}$.
\\Adding all sum and multiplications together, $\frac{n^2+3n}{2}+n+1=O(n^2)$
\\The time complexity in this polynomials is $O(n^2)$
\\b).first, we prove the base case, for $n=1$, $z_n=a_1x_0+a_0=p_n(x_0)$.
\\we need to prove the case for $n+1$ is also true. $p_{n+1}(x_0)=p_n(x_0)+a_{n+1}x^{n+1}=z_n+a_{n+1}x^{n+1}=z_{n+1}, \ because \ z_n=p_n(x_0)$
\\so, we can infer that $z_n=p_n(x_0)$, for all n, and the Horner's rule output the $p_n(x_0)$.
\\c). In the Horners algorithm, it go through the loop form $n-1$ to 0. In each loop, it has two steps. $\sum _{i=n-1}^0\:2=O(n)$
\\The time complexity of Horners algorithm is $O(n)$

\subsection{problem4}
a). $T(n)=2T(\frac{n}{2})+\sqrt{n}$
\\The branching factor is 2, and the number of subproblems at depth k are $2^k$, the size of subproblems at depths k is $2^k*\sqrt{\frac{n}{2^k}}$. The total depth is $\frac{n}{2^k}=1, \  k=log_2(n)$. The total running time is $\sum _{k=1}^{log\left(n\right)}\:\sqrt{\frac{n}{2^k}}=\Theta(n)$


b).The branching factor is 2, and the number of subproblems at depth k are $2^k$, the size of subproblems at depths k is $2^k$. The total depth is $\frac{n}{3^k}=1, \  k=log_3(n)$. The running time is $\sum _{k=0}^{log_3\left(n\right)}\:2^k=\Theta (n^{log_3^2})$.

c).The branching factor is 5, and the number of subproblems at depth k are $5^k$, the size of subproblems at depths k is $\frac{5^k*n}{4^k}$. The total depth is $\frac{n}{4^k}=1, \  k=log_4(n)$. The running time is $\sum _{k=0}^{log_4\left(n\right)}\:\frac{5^k\cdot \:n}{4^k}=\Theta(n^{log_4^5})$.

d).The branching factor is 7, and the number of subproblems at depth k are $7^k$, the size of subproblems at depths k is $n$. The total depth is $\frac{n}{7^k}=1, \  k=log_7(n)$. The running time is $\sum _{k=0}^{log_7\left(n\right)}n\:=\Theta(n*log_7^n)$.

e).The branching factor is 9, and the number of subproblems at depth k are $9^k$, the size of subproblems at depths k is $n^2$. The total depth is $\frac{n}{3^k}=1, \  k=log_3(n)$. The running time is $\sum _{k=0}^{log_3\left(n\right)}n^2=\Theta(n^2*log_3^n)$.





\end{document}