\documentclass[11pt]{article}
\usepackage{graphicx}
\graphicspath{ {C:/Users/yedkk/Desktop/CS465/HW1} }
\begin{document}
\section{Homework 3}
Name: Kangdong Yuan
	
\subsection{probelm1}
a).I did not work in a group.
\\b).I did not consult without anyone my group members
\\c).I did not consult any non-class materials.

\subsection{problem2}
Master’s theorem, we can split the recurrence into $T(n)=aT(\frac{n}{b})+(n^k*log^p(n))$ \\
a). $T(n)=11T(\frac{n}{5})+13n^{1.3}, \ a=11, \ b=5, \ k=1.3, \ p=0$\\
$log_{5}(11)=1.489>1.3=k$ Then the running time is $ T(n)=\Theta(n^{log_{5}(11)})$ \\
\\
b). $T(n)=6T(n/2)+n^{2.8}, \ a=6, \ b=2, \ k=2.8, \ p=0$\\
$log_{2}(6)=2.58498<2.8=k$ Then the running time is $ T(n)=\Theta(n^{log_{2}(6)})$ \\
\\
c). $T(n)=5T(n/3)+log^2(n), \ a=5, \ b=3, \ k=0, \ p=2$\\
$log_{3}(5)=1.464>0=k$ Then the running time is $ n^{log_{3}(5)}=\Theta(n^{1.464})$ \\
\\
d).$T(n)=T(n-2)+log(n)$, we can unfold it $T(n)=T(n-4)+log(n-2)+log(n)=T(n-6)+log(n-4)+log(n-2)+log(n)$ \\ Finally, if n is a even number $ T\left(n\right)=T\left(0\right)+\sum _{k=1}^{\frac{n}{2}}log\left(2k\right)$ \\
if n is a odd number, $T\left(n\right)=T\left(1\right)+\sum _{k=1}^{\frac{n}{2}}log\left(2k+1\right)$ \\
\\ The running time of \\ $Min (T\left(1\right)+\sum _{k=1}^{\frac{n}{2}}log\left(2k+1\right),T\left(0\right)+\sum _{k=1}^{\frac{n}{2}}log\left(2k\right) )  \leq T(n) \leq Max(T\left(1\right)+\sum _{k=1}^{\frac{n}{2}}log\left(2k+1\right),T\left(0\right)+\sum _{k=1}^{\frac{n}{2}}log\left(2k\right))$ \\
$T(n)=O(n*log(n))$ and $T(n)=\Omega(n*log(n))$,  
So, for all n, the running time of $T(n)$ is $\Theta(n*log(n))$

\subsection{problem3}
We assume the n is the power of 2. Find the $A[i]=i$, first we know it is a sorted array, so we can use this feature to find when is the $A[i]=i$. \\
By using the divide and conquer, first divide this array by 2, and find the mid item. Because there are $n+1$ items in this array (the start index of this array is 0), the mid item in this array is $A[\frac{n}{2}]$ then check whether $A[\frac{n}{2}]=\frac{n}{2}$, if the conditions are true, we return the true. \\ If the conditions are not true, \\ if $A[\frac{n}{2}]>\frac{n}{2}$ we need do the same search in lower interval which from 0 to $\frac{n}{2}$. \\ But if $A[\frac{n}{2}]<\frac{n}{2}$ we need do the same search in upper interval which from $\frac{n}{2}$ to n. \\ We do this recurrence search process until we find the $A[i]=i$, or the search interval becomes 1.\\
The recurrence function is $T(n)=T(\frac{n}{2}+1)$, then we solve the running time by Master’s theorem. $a=1, \ b=2, \ k=0, \ p=0$, $log_2(1)=0=k, p>-1$ so we can know that running time is $O(n^k*log^{p+1}(n))=O(log(n))$

\subsection{problem4}
For each array, there are $n$ items in this array (the index of this array start from 1), and there are m different values in this array. \\
The sorted algorithm for this list is go through all the items in the array, we create m different new array to to store the values for each items. For example, the a new array store only value 5, then we go through all the items in array, we put all items = 5 into this array. Then, after put all items into new arrays, we compare the value in each sub-array by insertion sort. Finally, after sort each sub-array, we combine all sub-array with wanted order.  \\ We analysis the running time of this algorithm, when we go through all the items in array it take O(n), when we sort M sub-array by insertion sort, the average running time is O(M) (if the value in array is uniformly distributed). The total running time is O(n+M). \\
For small M, the running time is also O(n+m), the lower bound is cannot be the $\Omega(nlog(n))$






\end{document}